\documentclass[12pt]{article}
\usepackage[T1]{fontenc}
\renewcommand{\contentsname}{Spis treści}
\author{Franciszek Magiera, Dawid Lech}
\title{Wykorzystanie algorytmu genetycznego w rozwiązywaniu problemu QAP.}
\date{09.01.2019}
\begin{document}
\maketitle
\newpage
\tableofcontents
\newpage
\section{Wstęp}
Naszym zadaniem było znalezienie takiego rozmieszczenia fabryk, znając przepływ materiałów między nimi oraz odległości pomiędzy odpowiednimi miejscami, aby energia potrzebna do transportu materiałów była minimalna. Jest to przypadek NP-trudnego problemu QAP. Rozwiązania problemu poszukiwaliśmy przy pomocy algorytmu genetycznego.
\section{Model matematyczny}
Mamy dany zbiór n fabryk oraz n dostępnych na budowę miejsc, które numerujemy od 1 do n. Dana jest również macierz przepływu $n\times n$, której element w i-tej kolumnie i j-tym wierszu oznacza ilość transportowanych materiału z fabryki oznaczonej numerem i do fabryki oznaczonej numerem j oraz symetryczną macierz odległości pomiędzy poszczególnymi miejscami, również o wymiarach $n\times n$, której element w i-tej kolumnie i j-tym wierszu oznacza odległość miejsca oznaczonego numerem i od miejsca oznaczonego numerem j (w obydwu przypadkach $i,j \in [1,n]$). Rozwiązania problemu szukamy w postaci n-elementowej permutacji, gdzie element na i-tym miejscu o wartości j oznacza przyporządkowanie fabryki o j-tym numerze na miejsce o i-tym numerze. Oznaczając macierz przepływu materiałów pomiędzy fabrykami przez W, macierz odległości pomiędzy dostępnymi miejscami przez D oraz permutację będącą rozwiązaniem poprzez s, funkcję celu możemy zapisać w postaci:
\begin{equation}
\sum_{i=1}^{n}  \sum_{j=1}^{n}D_{i,j}W_{s(i), s(j)}
\label{Funkcja celu}
\end{equation}
\end{document}

